\documentclass[a4paper]{article}
\usepackage{asymptote}

\input{preamble.tex}
\input{letterfont.tex}
\input{macros.tex}

\chead{ANTH 230: Gender Across Cultures}

\begin{document}
\section{Biological Sex} % 1/29

\subsection{The 5 Markers of Biological Sex}
Typically 5 markers used to determine sex
\begin{itemize}
  \item Chromosomes
  \item Hormones
  \item Inner genetalia
  \item Outer genetalia
  \item Secondary sexual characteristics
\end{itemize}

\subsection{Sex Development}
\begin{itemize}
  \item Chromosomes \begin{itemize}
    \item XY (typically male)
    \item XX (typically female)
  \end{itemize}
  \item All embryos begin with indistinguishable sex and have potential to become a variety of sexes
  \item Fetus starts to form sexual characteristics around age 5-7 weeks
\end{itemize}

\subsection{Chromosomal Variance}
\begin{itemize}
  \item Not all chromosomal compositions are XX or XY
  \item Over 70 chromosomal combinations documented
  \item Roughly 1 in every 426 humans have a chromosomal makeup that does not conform to XX/XY binary \begin{itemize}
    \item Statistically $ \sim $125 students at Purdue
    \item Likely higher than 1/426 odds; not many people get their chromosomes tested
  \end{itemize}
  \item Examples: \begin{itemize}
    \item XXY or XXXY (Klinefelter's Syndrome)
    \item XO (Turner's Syndrome)
  \end{itemize}
\end{itemize}

\subsection{Hormones}
\begin{itemize}
  \item Glands secrete combination of the hormones testosterone, estrogen, and progesterone
  \item 2 essential washing of hormones - in utero and at puberty
  \item Different combinations = different effects on the body
  \item Interaction between chromosomes and hormones determined development of physical body
  \item Variety of physically diverse bodies
\end{itemize}

\subsection{Hormones and the Life Course}
\begin{itemize}
  \item Horomone production varies over the life course
  \item Endocrine system influences body and brain functioning
  \item Influenced by external factors \begin{itemize}
    \item Environment
    \item Diet
    \item Social roles
  \end{itemize}
\end{itemize}

\subsection{Chromosomes + Hormones}
Chromosomes and hormones encourage the development of primary and secondary sexual characteristics
\begin{itemize}
  \item Primary (M): testes, penis, scrotum, prostate,
  \item Primary (F): ovaries, fallopian tubes, uterus, vagina
  \item Secondary (F): breast development, wider hips, higher body fat (esp. in hips, thighs, butt)
  \item Secondary (M): deepening of voice, broader shoulders, increased muscle mass, growth of body hair
\end{itemize}

\subsection{Transgressive Bodies}
\begin{itemize}
  \item Rougly 2\% of the population is non-binary
  \item Medicalization of the body leads normative assumptions about the body and categorizes some as `normal' and others as `abnormal'
  \item Frequently we `correct' bodies that do not fit the binary \begin{itemize}
    \item Infanticide
    \item Murder
    \item Sexual assignment surgery
    \item Hormone Therapy \begin{itemize}
      \item Over assignment to female
      \item ``Easier to dig a hole than to build a hole''
    \end{itemize}
  \end{itemize}
\end{itemize}
\end{document}