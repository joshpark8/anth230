\documentclass[a4paper]{article}
\usepackage{asymptote}

\input{preamble.tex}
\input{letterfont.tex}
\input{macros.tex}

\chead{ANTH 230: Gender Across Cultures}

\begin{document}
\section{Gender 101 3} % 02/19
\subsection{Gender and Hegemony}
\begin{itemize}
  \item Gendered Subject Personalities - Subject positions often offer multiple `ways of being' within each category \begin{itemize}
    \item Ex: masculinities
  \end{itemize}
  \item Each subject positionality will be ranked (and valued) based on its adherence to the hegemonic (dominant/privileged) notion of that gender (Hegemonic Gender)
  \item Subject positionalities are typically ideals rather than representative of concrete real world dynamics \begin{itemize}
    \item Often shape our mental maps of reality, whence shaping legal and otherwise concrete societal frameworks
  \end{itemize}
  \item Individuals may occupy one subject position while exhibiting characteristics of another \begin{itemize}
    \item Ex: `tomboy', `dyke', `queer'
  \end{itemize}
  \item Hegemonic Gender intersect with Hegemonic Positionalities/Categories
\end{itemize}

\subsection{Gender and Social Stratification} \begin{itemize}
  \item Mental Maps of Reality overlap and create associations
\end{itemize}
\begin{itemize}
  \item Sherry Ortner - ``Is Female to Male as Nature is to Culture?''
  \begin{itemize}
    \item Females reproduce offspring and therefore produce `naturally'
    \item Males must pproduce `artifically' (culturally)
    \item Culture dominates Nature $\imp$ Males (men) dominate Females (women)
  \end{itemize}
\item Critiques
\begin{itemize}
    \item Overlooks cultures with more than gender binaries
    \item Utilizes a Western lens to assess `power and agency'
    \item Ignores the massive impact that European colonization had on restructuring cultural gender categories and ideologies
  \end{itemize}
\end{itemize}

\subsection{Gender and Kinship}
\begin{itemize}
  \item Kinship - A cultural system of defining who is related to whom and what obligations members of a kin network have to each other
  \item Kin groups/family - The most basic political, organizational, and economic (resource management) unit in society
  \item Patterns of Kinship influence social interactions, access to resources, and larger social organization and patterns
\end{itemize}

\subsection{Descent and Lineage}
\begin{itemize}
  \item Systems of determining who is related to whom
  \item Unilineal \begin{itemize}
    \item Matrilineal
    \item Patrilineal
  \end{itemize}
  \item Bilateral
\end{itemize}
\end{document}