\documentclass{article}
\usepackage{asymptote}

\input{preamble}
\input{letterfont}
\input{macros}

\chead{ANTH 230: Gender Across Cultures}

\begin{document}
\section{Gender 101} % 02/12

\subsection{What is Gender?}
\begin{itemize}
  \item Gender is culturally constructed and historically informed
  \item Gender is learned - culturally transmitted
  \item Concepts of gender differ cross-culturally
  \item Concepts of gender are constested within/across cultures
  \item Gender roles are socialized from an early age
\end{itemize}

\subsection{Babies and Gender}
\begin{itemize}
  \item Gender stereotyping begins as early as 3 months old
  \item Parents may respond to crying children differently depending on assumed (from external genetalia) gender
  \item These discrepancies can have a profound impact on neurological development and can leave traces on genes
  \item Fathers found to see less value in letting children participate in activities typically associated with the `other' gender \begin{itemize}
    \item Indicates boys are taught to be more restrictive in the toys they play with and the activities they partake in than girls are
    \img{img/10-dads.png}{.8}
  \end{itemize}
\end{itemize}

\subsection{Gendered Clothing}
\begin{itemize}
  \item Skirts vs. Pants \begin{itemize}
    \item Differs across cultures
    \item Differs within a single culture across time
  \end{itemize}
  \item The great switch of 1920's \begin{itemize}
    \item Gendered clothing was invented by marketers/manufacturers in the 1920's to make people want to buy more clothes for different gendered children
  \end{itemize}
\end{itemize}

\subsection{Gendered Speech}
\begin{itemize}
  \item Study on communication within a business meeting \begin{itemize}
    \item Women speaking 50\% of the time: \tit{BOTH} men and women percieved women to dominate the conversation
    \item Women speaking 30\% of the time: \tit{BOTH} men and women percieved women to be equally participating
  \end{itemize}
\end{itemize}

\subsection{Gender and Power}
\begin{itemize}
  \item \tit{Patriarchy} - a system of societal and institutional organization that privileges men and masculinity
  \item \tit{Sexism} - prejudice and/or discrimination against someone on the basis of gender \begin{itemize}
    \item Reverse sexism does \tit{not} exist (in the United States)
    \item Patriarchy is inherently a system of power
    \item Women and femininity is devalued
    \item Thus, sexism can only occur to the devalued group
  \end{itemize}
\end{itemize}

\subsection{Gender Identity}
\begin{itemize}
  \item How we feel and see ourselves and how we want others to see us
  \item Difficult to develop gender identity that exists outside of the confines of the ideologies and terminologies of the society you exist in
\end{itemize}

\subsection{Gender Performance}
\begin{itemize}
  \item How we express or `do' gender
  \item Performance $ \neq $ fake
  \item Performace refers to individual outward expression
\end{itemize}

\subsection{Gender Embodiment}
\begin{itemize}
  \item Routine repetitious actions become emedded within us
  \item Becomes `naturalized' or performed unconsciously
\end{itemize}

\subsection{Gender Ideals/Stereotypes}
\begin{itemize}
  \item Concepts of gender most often represent cultural ideals rather than reality or how people actually behave
  \item Gender is constantly (re)produced and contested
\end{itemize}

\subsection{\ Intersectionality}
\begin{itemize}
  \item Interlocking systems of power that account for the multiple social positions we occupy
  \item \tit{Cannot} discuss gender without talking about other oppressive categories such as \begin{itemize}
    \item Nationality
    \item Race
    \item Ability
    \item Sexual Orientation
    \item Class
  \end{itemize}
  \item Ex: ``Second Wave Feminist Movement'' \begin{itemize}
    \item Called a feminist movement but was strictly for upper class straight white women with U.S. citizenship
    \item The subset of women above wanted to abolish the idea that working outside of the home was strictly for men
    \item Lower class/immigrant women had already been working outside of their homes in order to make enough money for their household, taking jobs in other peoples homes or in factories
  \end{itemize}
\end{itemize}

\subsection{\ Third Genders}
\begin{itemize}
  \item Native American -- Two Spirit
  \item Samoan -- Fa'afafine
  \item South Asian -- Hijra
\end{itemize}

\end{document}