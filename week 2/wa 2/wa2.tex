\documentclass{article}

\input{preamble}
\input{letterfont}
\input{macros}

\fancyhead[L]{\tbo{Josh Park\\ANTH 230}}
\fancyhead[C]{\tbo{Writing Assignment 2}}
\fancyhead[R]{\tbo{Spring 2025\\Prof. Wirtz}}

\begin{document}
The first video, `Humanthropology', defines and lays out the basic framework of the field of anthropology, emphasizing in particular the critical differences between it and the adjacent field of archeaology.
Anthropologists have long worked to answer many of the `big questions' about the human condition, such as how we evolved, how we interact and communicate, and how the institutions that hold societies together are formed.
Contemporary anthropologists lend their expertise to a variety of other disciplines, including medicine, pedagogy, agriculture, and social justice.
Commonly considered to be the earliest historian, the ancient Greek philosopher Herodotus laid the groundwork necessary for anthropology when he documented the various cultures and traditions of people while travelling across the Persian empire amidst the Greco-Persian Wars around 500BCE.
This could be considered an early iteration of ethnography, the systematic study of people and cultures, in which the observer places themselves in the shoes of an indigenous person as a means of immersing themselves in the culture.
However, ethnography is closely tied with the outdated ideas of scientific racism, as ethnographers would commonly document and measure the physical characteristics of the people they studied.
While anthropology has strong roots in racism, it is important that we acknowledge how it has impacted the field so that future anthropolgists can work to avoid similar mistakes.

The second video, `What is Culture?' explains why and how the term ``culture'' is so hard to define.
While probably everyone knows what culture \tit{is}, it is such a fundamental part of the human experience that it is surprisingly abstract.
Rather than trying to restrict it to a single definition, culture can be thought of as a process that is
a) learned, b) shared, c) symbolic, d) integrated, e) adaptive, and f) performed.
These aspects of culture allow it to express itself in the many ways we see constantly in the world around us.
Consider a child who is genetically Chinese and adopted into a white American family as a baby.
The family would then \tul{share} their culture with the child by \tul{performing} cultural rituals or traditions \tul{symbolic} of the values of that culture, and the child would \tul{learn} and \tul{integrate} these practices to their own life in order to navigate the society in which they live, perhaps \tul{adapting} it to fit themselves better along the way.
In this way, culture is constantly evolving and changing with the people who practice it, and it is this dynamism that makes it so difficult to pin down.
\end{document}
