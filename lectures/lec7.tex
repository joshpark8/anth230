\documentclass[a4paper]{article}
\usepackage{asymptote}

\input{preamble.tex}
\input{letterfont.tex}
\input{macros.tex}

\chead{ANTH 230: Gender Across Cultures}

\begin{document}
\setcounter{section}{6}
\section{Biological Sex 2} % 2/03
\subsection{Mental Maps of Reality}
\begin{itemize}
\item How we caegorize the world arounnd us
\item How we assign value to different categories
\end{itemize}

\subsection{Categories and Dualisms}
\begin{itemize}
\item In order to understand and make sense of the world, we create categories
\item By definition these are exclusionary and reductive, and thus rarely represent `reality'
\item Categories often come in binaries and set in opposition to one another (dualisms) \begin{itemize}
\item Nature/Culture
\item Sex/Gender
\item Homo/Hetero
\item Mind/Body
\end{itemize}
\end{itemize}

\subsection{Cultural Narratives of the Body}
\begin{itemize}
\item Our culural beliefs about similarities/differences influence how we see the body and understand sex
\end{itemize}

\subsection{Sexing the Body}
\begin{itemize}
\item Gender ideology - set of culutural ideas about the essential character of different genders \begin{itemize}
\item Sex and gender as binary
\item Sex and gender as synonymous
\end{itemize}
\item Gender ideology shapes concepts of biological sex and how the body works
\item Popular and medical notions of biological sex often reflect our culutural beliefs rather than reality
\item Scientific terminology and medicalization change the way we culturally understand the body
\end{itemize}

\subsection{The Egg and the Sperm}
\begin{itemize}
\item Traditional notion of conception is based on stereotypical male-female gender roles
\end{itemize}

\subsection{Biopower}
\begin{itemize}
\item The processes by which regimes of authority produce knowledge on human life and the power these systems of knowledge have on social institutions and individual lives
\end{itemize}
\end{document}