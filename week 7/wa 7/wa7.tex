\documentclass{article}

\input{preamble}
\input{letterfont}
\input{macros}

\fancyhead[L]{\tbo{Josh Park\\Prof. Wirtz}}
\fancyhead[C]{\tbo{ANTH 230: Gender Across Cultures\\Writing Assignment \thesection}}
\fancyhead[R]{\tbo{Spring 2025\\Page \thepage}}

\begin{document}
\setcounter{section}{7}
\begin{prompt}
  In the video, Joe Ehrmann mentions 3 different lies that young boys in the US learn about masculinities. Give at least two examples of these `lies' that you have witnessed/experienced in your own life. Explain which of the three lies each example is tied to and why.
\end{prompt}
Joe Erhmann's 3 lies are \begin{enumerate}
  \item The Ballfield Lie - A man's worth is intrinsically tied to his physical and athletic prowess
  \item The Billfold Lie - A man's worth is intrinsically tied to his financial success
  \item The Bedroom Lie - A man's worth is intrinsically tied to his sexual conquests
\end{enumerate}
I have seen these play out countless times in my life, but the two specific examples I chose are related to the Billfold and Bedroom Lies, respectively.

My example for the Billfold lie is when I was talking with some classmates about our internship prospects for the upcoming summer.
As in many STEM fields, some internships for computer science can pay interns incredibly well.
In this conversation there was a classmate who was very good at coding who got a very high paying offer at a quantitative finance firm, and it was obvious by the way he kept asking people how much their respective positions would pay that he just wanted to flex his wage over the rest of us.
It put a big damper on the conversation right away as nobody else wanted to share their experiences just to get looked down upon by a peer.

For the Bedroom Lie, I was hanging out and studying with a friend from high school during my second year at Purdue, when he suddenly asked me how many `bodies' I had `caught' since starting college.
This caught me off guard because while I was not unfamiliar with that type of langauge, it wasn't something most of my friends or I would use, nor was it something I had had directed at me.
Personally, I value emotional connections a lot more than physical flings, so I have a lot of pride in how selective I am with my relationships and did not think much of admitting to him that I had not slept with any girls since starting school.
Without saying it explicitly, it was obvious he looked down on me for what I had said, as he went on to brag about how many people he was sleeping with.

\begin{prompt}
  Explain what ``the mean team'' is. How do dynamics such as this negatively impact young boys as they grow up/mature?
\end{prompt}

``The mean team'' refers to the collective pressure boys face from peers and societal expectations to adopt aggressive, dominant, or unemotional traits.
In practice, ``the mean team'' can manifest as cliques that shame those who show vulnerability or who deviate from rigid masculine norms.
The group dynamic rewards behaviors like put-downs, bullying, or emotionally distant, stoic, and aloof attitudes, which creates an environment unfriendly towards any empathy and emotional openness.

These dynamics can negatively affect young boys as they mature by severely limiting their emotional range and stunting their interpersonal relationships.
When boys learn early on that showing kindness or vulnerability might make them targets of ridicule, they often repress genuine emotions.
This can lead to higher rates of stress, difficulty forming healthy relationships, and, in some cases, a turn toward destructive coping mechanisms.
In adulthood, these behaviors are often self-perpetuating and contribute towards reinforcing cycles of aggression, isolation, and an inability to seek help when needed.
In particular, men who have been effectively bullied into submitting to this emotional framework may push those values onto their sons, who may go on to perpetuate the cycle even further.

\begin{prompt}
  Using specific examples from the video and class content (lectures, readings, discussion, etc), explain how culture and power influence gendered experiences. How do these concepts intersect with concepts like violence (physical, structural, psychological) and agency? Write at least a 250 word response.
\end{prompt}

Culture, power, and gendered experiences are deeply intertwined in shaping both individual identity and broader social structures.
We know that gender is not an inherent, fixed trait, but rather it is a system of overlapping categories constructed through social processes and historical context.
From birth, individuals are immersed in expectations that guide behavior, thought, and interaction.
For example, even very young infants are often treated differently based on their perceived gender, with caregivers responding in distinct ways to their cries or play.
Such early socialization creates foundational mental maps that influence identity development and later behaviors.

Gender is actively performed in everyday life, and these performances are learned and reinforced through repeated cultural practices.
Individuals express their gender through actions like dressing, speaking, and managing emotions.
These behaviors are not random but are deeply rooted in societal norms that dictate what is appropriate for different genders.
A clear example of this is manspreading—a behavior where some men adopt a wide seating posture in public spaces.
This is not an inherent or natural physical difference but a learned behavior that reinforces dominant masculine norms.
Manspreading, along with other gendered behaviors, exemplifies how cultural practices become embodied habits that signal and perpetuate specific gender identities.

Power dynamics are central to the construction and reinforcement of gender norms.
Patriarchal systems privilege traits associated with masculinity while devaluing those linked to femininity, thus creating a social hierarchy that influences access to resources and opportunities.
Mechanisms of biopower—where institutions regulate and control bodies through norms and standards—serve to further enforce these hierarchies.
Moreover, gender intersects with other social identities such as race, class, nationality, and sexual orientation, resulting in complex patterns of privilege and oppression.
For instance, the compounded discrimination experienced by women of color, often discussed as misogynoir, highlights how overlapping systems of power shape lived experiences.
\end{document}
