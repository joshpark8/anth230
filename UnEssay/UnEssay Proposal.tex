\documentclass{article}

\input{preamble}
\input{letterfont}
\input{macros}

\fancyhead[L]{\tbo{Josh Park\\Prof. Elizabeth Wirtz}}
\fancyhead[C]{\tbo{ANTH 230: Gender Across Cultures\\UnEssay Topic Proposal}}
\fancyhead[R]{\tbo{Spring 2025\\Page \thepage}}

\setlength{\parindent}{15pt}

\begin{document}
For my project, I plan to leverage my technical expertise in mathematics/computer science to explore gender inequality through data visualization.
In particular, I am interested in analyzing disparities in education, wages, and political representation across different cultures.
I plan to use Anthrosource to find relevant studies that I can pull data from, and to set up automated API calls to real-world datasets from sources like the World Bank and UN Women to present a compelling argument about how structural and cultural factors contribute to these inequalities.
My hope is to highlight how gender roles differ across time and space, and how those disparities impact individuals as well as societies as a whole.
My analysis will be rooted in anthropological inquiry, examining the statistics not just in an isolated environment, but in the context of the historical and cultural forces that have shaped these inequalities.

To achieve this, my idea is to develop an interactive dashboard using the programming language Python, that will allow users to play with the parameters and see how the data reflects their inputs.
The visualization will include elements such as colored maps, bar charts, and time-series graphs, making it easy to compare different regions and time periods.
By giving users the ability to filter and interact with the data, my goal is to make gender inequality a more tangible and thought-provoking issue that can not be easily ignored.
This project will go beyond displaying raw numbers, and I hope that by incorporating qualitative insights to contextualize the trends observed, I will be able to emphasize the importance of thinking anthropologically about gender disparities.
\end{document}
