\documentclass{article}

\input{preamble}
\input{letterfont}
\input{macros}

\fancyhead[L]{\tbo{Josh Park\\ANTH 230}}
\fancyhead[C]{\tbo{Writing Assignment 3}}
\fancyhead[R]{\tbo{Spring 2025\\Prof. Wirtz}}

\begin{document}

\begin{enumerate}
  \item What does it mean to be `intersex'? How do medical professionals talk about intersex, and how does that compare with how intersex people talk about their own bodies? \begin{itemize}
    \item Being intersex refers to a range of natural variations in sex characteristics (such as chromosomes, gonads, hormones, or genitalia) that do not fit normative definitions of male or female. Medical professionals often use clinical terms like ``disorders of sex development'', framing intersex traits as conditions that need to be ``managed'' or ``corrected'' through medical intervention. This perspective tends to emphasize binary sex classification and often leads to unconsentually administered surgical or hormonal interventions at an early age. However, intersex individuals and advocates often reject the pathologizing language of medicine, preferring terms like ``variations of sex development''. Many intersex activists argue that their bodies are simply natural variations of human biology and that medical interventions should not be performed without informed consent.
  \end{itemize}
  \item How have cultural beliefs about biological sex shaped common medical and social approaches to intersex people? \begin{itemize}
  \item Western beliefs heavily reinforce a binary understanding of biological sex (male/female), which has influenced medical and legal policies regarding intersex people. Historically, medical professionals sought to assign intersex infants a definitive sex as quickly as possible, often through irreversible (and possibly damaging) surgeries, to fit societal expectations. Intersex people are faced with harsh stigmas and pressure to conform to binary gender norms, often leading to irreversible medical trauma and identity erasure. Moreover, the legal system often fails to recognize intersex identities, which reinforcing policies that mandate classification as male or female.
\end{itemize}
  \item Give an example of two of the ways intersex people have used the internet to connect with each other and share information. Why are these forms of connection important? \begin{itemize}
  \item In recent decades, intersex activists have been able to leverage of online forums, blogs, and social media to share experiences, educate others, and challenge medical and societal narratives. Organizations like Intersex Justice Project and InterACT use digital platforms to advocate for intersex rights and legal protections.	Online communities, such as the Intersex Reddit forums or private Facebook groups, provide safe spaces for intersex individuals to share their stories without fear of stigma or medical coercion. These connections are extremely crucial because intersex individuals often feel isolated and are misinformed about their own bodies, told that there is a fundamental problem with themselves. Through these platforms, intersex individuals can share their experiences with others who have gone through similar hardships and find solidarity in numbers. The internet allows them to reclaim their narratives, resist harmful medical practices, and build collective power.
  \end{itemize}
  \item Relate this video to other course content. What does this video on intersex people tell us about what we are learning this week?
  \begin{itemize}
  \item This video illustrates how biological sex is far more complex than a simple male-female binary, aligning with the discussion of chromosomal variance, hormonal influence, and sex development. The existence of over 70 documented chromosomal variations challenges the assumption that all humans fit neatly into XX or XY categories. Since many people never have their chromosomes tested, the actual number of individuals who do not conform to the binary is likely much higher. Historically, intersex individuals have been subjected to surgical and hormonal interventions designed to ``correct'' their bodies, often without their consent. This is indicative of broader societal discomfort with ambiguity, where bodies are either classified as ``normal'' or ``abnormal'' based on cultural expectations rather than natural biological diversity. The frequent over-assignment of intersex infants to female, based on the idea that it is ``easier to dig a hole than to build one,'' demonstrates how medical decisions are shaped by convenience and cultural biases rather than the needs of the individual.
  \end{itemize}

\end{enumerate}

\end{document}
