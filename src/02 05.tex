\documentclass[a4paper]{article}

\input{preamble.tex}
\input{letterfont.tex}
\input{macros.tex}

\chead{ANTH 230: Gender Across Cultures}

\begin{document}
\section{Biological Sex 3} % 02/05

\subsection{Gendered Physiological Discrimination}
\begin{itemize}
\item Ewa Klobukowska \begin{itemize}
\item Polish sprinter; set multiple world records in the 1960s
\item "Failed a gender test" and was banned/stripped of all all records
\item Gave birth to a son a year later
\end{itemize}
\item Ordering an disciplining of bodies on hormones, chromosomes and other biological processes
\item Explores this concept through `Sex Verification Testing' in elite sports
\item Example of Biopower
\end{itemize}

\subsection{Sex, Gender, Race, and Imperialism}
\begin{itemize}
\item Current International Olympic Committee rules: \begin{itemize}
\item `Suspicion-Based Testing' vs `Universal Testing'
\item DNA Chromosome Tests
\item Hormone Level Tests
\end{itemize}
\item Types of people considered `suspicious' are not treated equally \begin{itemize}
\item People identifying as women of color targeted almost exclusively
\end{itemize}
\item Bodies targeted for sex suspicion and testing tend to come from marginalized or `othered' populations
\end{itemize}

\subsection{Intersectionality}
\begin{itemize}
\item We are all a combination of various social identities and positions \begin{itemize}
% \item Religion (Christian vs. non-Christian)
\item Sexual orientation (Straight vs. LGBTQ+)
% \item Age (middle-aged vs. seniors/youth)
\item Culture (Western vs. non-Western)
\item Education (post-secondary vs. no formal education)
\end{itemize}
\item These positionalities intersect to create unique social groups and experiences, including privilages/oppressions
\item Ex: Gender (sexism) + race (racism) = ``misogynoir'' \begin{itemize}
\item Unique form of oppression experienced by female POCs
\end{itemize}
\end{itemize}



\end{document}