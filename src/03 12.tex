\documentclass[a4paper]{article}

\input{preamble.tex}
\input{letterfont.tex}
\input{macros.tex}

\chead{ANTH 230: Gender Across Cultures}

\begin{document}
\section{Masculinities} % 03/12
\subsection{Becoming a Woman}
\subsubsection{Early anthropologists on Womanhood}
\begin{itemize}
  \item Relatively fewer `rites of passage' for girls to become women
  \item Mostly associated with biological processes like menses (onset of menstruation) or giving birth (becoming a mother)
  \item Sometimes associated with related processes like marriage (becoming a wife)
  \item Femininity often associates with similar characteristics to childhood \begin{itemize}
    \item Submissive, dependent, innocent, ignorant
  \end{itemize}
  \item Thus, there is less pressure to distance oneself from childhood to establish adult womanhood
\end{itemize}

\subsubsection{Contemporary anthropologists}
\begin{itemize}
  \item Recognize there is a variety of types of rites of passage for girls into womanhood that are not related to reproduction
\end{itemize}

\subsection{Becoming a Man}
\begin{itemize}
  \item `Manhood' like `Womanhood' involves socially constructed concepts that demarcate transition from child to adult social roles
  \item Transition from boy to man often considered more challenging because there is a lack of significant biological events that would demonstrate `manhood'
  \item The construction of manhood and masculinity often involves a distancing fom attributes considered feminine and/or associated with femininity
  \begin{itemize}
    \item `Boys don't cry'
    \item `Be a man'
  \end{itemize}
\end{itemize}

\subsection{Rites of passage}
\begin{itemize}
  \item Ritualistic
  \item Guides change of status in society
  \item Overseen by elders (authority)
  \item Predetermined sequence of events
\end{itemize}
\subsubsection{Three phases}
\begin{itemize}
  \item Separation
  \item Liminality
  \item Incorporation
\end{itemize}
\end{document}